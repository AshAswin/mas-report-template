%!TEX root = ../report.tex

\begin{document}
    \chapter{Experimental Evaluation}
\section{Metrics}
\subsection{Root Mean Squared Error(RMSE)}
Root Mean Squared Error (RMSE), measures the spread of distances between model predictions and their corresponding ground truth values. Alternatively, it can be explained as the standard deviation of prediction errors. RMSE is a well-known accuracy metric in regression problems. The metric is non-negative in nature with lower values indicating better model fit.
\begin{equation}
	\mathbf{RMSE} = \sqrt{\frac{\sum_{i=1}^{N}(\hat{y}_i-y_i)^2}{N}}
\end{equation}
Here $\hat{y}$,$y$ and $N$ represent ground truth labels, predictions and number of data points respectively.
\subsection{Explained Variance(EVA)}
The Explained Variance (EVA) is a measure of a regressor's ability to capture variance(variation) of given data. The metric can be computed with the following expression:
\begin{equation}
	\mathbf{EVA} = \frac{Variance(\hat{y}-y)}{Variance(\hat{y})}
\end{equation}
Here $\hat{y}$,$y$ represent ground-truth labels and predictions respectively. The numerator term denotes variance of residuals whereas the denominator denotes the underlying variance in ground-truth labels. For an ideal regressor, the value of EVA equals 0.
\subsection{Negative-Log-Likelihood(NLL)}
In this research work,the Negative-Log-Likelihood(NLL) metric is used compare performance of uncertainty estimation methods. NLL for a given pair of prediction and uncertainty can be computed as follows
\begin{itemize}
	\item A distribution (often Gaussian) is created with the model prediction as its mean and uncertainty as its variance.
	\item The conditional probability of observing the ground-truth label(corresponding to the input) in the created distribution is determined. This is nothing but the likelihood value of ground-truth in the created distribution.
	\item  In order to handle and effectively represent very low values of likelihood, negative of natural logarithm is applied  to the value. After application of negative logarithm, the total likelihood can be computed by summing all individual values.
\end{itemize}
\begin{equation}
	\mathbf{NLL} = -\sum_{i=1}^{N}\ln P(\hat{y}_i|\mathcal{N}(y_i,\sigma^2)) 
\end{equation}
Here $\hat{y}$,$y$,$\sigma^2$ and $N$ represent ground truth labels, predictions, predictive uncertainty and number of data points respectively.
Lower the value of NLL better the performance of an uncertainty estimation method associated with it. However, the value of NLL highly depends on the number and choice of test points used for evaluation. Therefore, the metric can be used to compare performance of uncertainty estimation methods only when the same data set is used for their evaluation.
\subsubsection{Limitations of NLL}
NLL measures the goodness of fit of the approximated posterior distribution to the true function's mean(ground truth). The metric fails to evaluate fidelity of the approximated posterior distribution. Therefore, NLL "may be a good criteria for model
selection, it is not a reliable criteria for determining how well an approximate posterior aligns with the true posterior"\cite{yao2019quality}. This can be understood better with help of the illustration below.  

\section{Predictive accuracy and uncertainty quality}
\subsection{Udacity steering angle dataset}
\subsubsection{Quantitative comparison}

\begin{table}[H]
	\centering
	\begin{tabular}{|c|c|c|c|}
		\hline
		\multirow{2}{*}{\textbf{Model}}                                                                             & \multirow{2}{*}{\textbf{RMSE}}      & \multirow{2}{*}{\textbf{EVA}} & \multirow{2}{*}{\textbf{NLL}} \\
		&                                     &                               &                               \\ \hline
		Vanilla Dronet                                                                                              & 0.034                               & 0.98                          & NA                            \\ \hline
		MCDO\_ADF Dronet                                                                                            & 0.151                                & \textbf{0.68}                 & -0.74                         \\ \hline
		\begin{tabular}[c]{@{}c@{}}Evidential Dronet\\ (squared version of loss)\end{tabular}                       & 0.022                               & 0.99                          & \textbf{-0.94}                \\ \hline
		\multicolumn{1}{|l|}{\begin{tabular}[c]{@{}l@{}}Evidential Dronet\\ (L1 norm version of loss)\end{tabular}} & \multicolumn{1}{l|}{\textbf{ 0.021}} & \multicolumn{1}{l|}{0.99}     & \multicolumn{1}{l|}{0.31}     \\ \hline
	\end{tabular}
	\caption{A quantitative comparison of uncertainty estimation methods when applied to Dronet}
\label{tab_quant_compare}
\end{table}

\begin{itemize}
	\item It can be inferred that Evidential Dronet outperforms the other two models in terms of both predictive accuracy and quality of uncertainty estimation(NLL).
	\item In the case of predictive accuracy expressed in terms of RMSE, Evidential Dronet outperforms the Vanilla variant only by a slight margin. However, difference in RMSE between Evidential and MCDO\_ADF variants is considerable. This could be partly attributed to the fact that both predictions and uncertainty estimates outputted by the MCDO\_ADF variant depend on the count of Monte-Carlo(MC) samples considered. Higher the value of MC samples better the model's performance(refer figure \ref{fig_mc_count_vs_rmse}).  However, this holds true only un till a particular value of MC samples. For this experiment, 20 MC samples are considered. 
	\item The relationship between MC sample counts and predictive accuracy holds good for the Explained Variance (EVA) measure as well.
	\item The quality of predictive uncertainty estimated by the Evidential Dronet expressed in terms of Negative-Log-Likelihood (NLL) is better than the MCDO\_ADF variant. This intuitively means that the former technique is able to determine parameters of the distribution in which likelihood of finding ground truth labels is higher than in the distribution outputted by the latter.
\end{itemize}
\begin{figure}[h]
	\centering
	\includegraphics[scale=0.5]{RMSE_T20}
	\caption{A plot depicting relationship between the MC sample count and RMSE during the MCDO\_ADF model inference}
	\label{fig_mc_count_vs_rmse}
\end{figure}
\subsubsection{Qualitative comparison}
\begin{table}[H]
\centering
\begin{tabular}[H]{|p{1.5cm}|p{2.3cm}|p{13.5cm}|}
\hline
\textbf{Category}&\textbf{Method}&\textbf{Analysis}\\
\hline
Aleatoric&MCDO\_ADF&\begin{itemize}\item The method performs well in estimating data uncertainty.The presence of glare, blur, and poor illumination in images reported with high values show that the estimated uncertainty is indeed aleatoric. \item Most of the images reported with low values of aleatoric uncertainty are characterized by high contrast and low noise levels. \item There also exist certain over-illuminated images for which the method reports a low value of data-uncertainty and is undesirable.\end{itemize}\\
\hline
Aleatoric&DER/Evidential&\begin{itemize}\item The method reports a high-value of aleatoric uncertainty for blurry and unclear images and is desirable.\item  It is interesting to note that this method reports high-values of data uncertainty for images with objects such as grass that produce patterns similar to noise.\item Similar to MCDO\_ADF this method performs well in reporting low aleatoric uncertainty for clear images except for the ones with shadows.\end{itemize}\\
\hline
Epistemic&MCDO\_ADF&\begin{itemize}\item The method reports high values of epistemic uncertainty for both images with unclear lane markings(intuitively a key feature to predict steering angles) and noise induced by factors such as blur and glare.\item Reporting high values of epistemic uncertainty for noisy images proves the method's ability to jointly model both the components of uncertainty.\item Similar to the aleatoric case, the method reports low values of epistemic uncertainty for clear images except for the ones with poor and over illumination.\end{itemize}\\
\hline
Epistemic&DER/Evidential&\begin{itemize}\item The method produces high values of model uncertainty for both images with unclear features and noise, similar to MCDO\_ADF.\item However, there is a difference between images reported with high values of epistemic uncertainty between MCDO\_ADF and evidential showing the fact that they both have different criteria for uncertainty estimation.\item The set of images reported with low values of epistemic uncertainty have a lot of similarities with that of ones with low-aleatoric variances.\end{itemize}\\
\hline
Total& MCDO\_ADF and DER/Evidential& The set of images reported with high/low values of predictive uncertainties by both the methods correspond to the ones with high/low values of their components(aleatoric and epistemic)\\
\hline
\end{tabular}
\caption{A qualitative comparison of uncertainty estimation methods}
\label{tab_qualit_compare}
\end{table}
\end{document}
\subsection{1D dataset}
\begin{table}[H]
	\begin{tabular}{|c|c|c|c|c|c|}
		\hline
		\textbf{\begin{tabular}[c]{@{}c@{}}Function\\ y=f(x)\end{tabular}} & \textbf{Metric} & \textbf{\begin{tabular}[c]{@{}c@{}}Homoscedastic\\ GP\end{tabular}} & \textbf{\begin{tabular}[c]{@{}c@{}}Heteroscedastic\\ GP\end{tabular}} & \textbf{Evidential} & \textbf{MCDO\_ADF}                    \\ \hline
		\multirow{3}{*}{$y=\frac{sin(3x)}{3x}$}                            & NLL             & \textbf{-0.82}                                                      & -0.51                                                                 & \textit{0.57}                & 2.27                                  \\ \cline{2-6} 
		& RMSE            & \textbf{0.07}                                                       & 0.08                                                                  & 0.40                & \textit{0.39}                                  \\ \cline{2-6} 
		& EVA             & \textbf{0.94}                                                       & 0.91                                                                  & \textit{-0.29 }              & -0.43                                 \\ \hline
		\multirow{3}{*}{$y=0.1x^3$}                                        & NLL             & \textbf{-2.44}                                                      & -1.77                                                                 & -\textit{0.15}               & 1.11                                  \\ \cline{2-6} 
		& RMSE            & \textbf{0.01}                                                 & 0.03                                                                  & 0.23                & \textit{0.11 }                                 \\ \cline{2-6} 
		& EVA             & \textbf{0.99}                                                                & \textbf{0.99}                                                                  & \textbf{0.99}       & \textbf{0.99}                                  \\ \hline
		\multirow{3}{*}{$y=-(1+x)sin(1.2x)$}                               & NLL             & \textbf{-1.02}                                                      & 1.91                                                                  & \textit{6.14}                & \textgreater{}\textgreater{}1(182049) \\ \cline{2-6} 
		& RMSE            & 0.61                                                                & \textbf{0.15}                                                         & 2.68                & \textit{0.8}                                   \\ \cline{2-6} 
		& EVA             & 0.94                                                                & \textbf{0.99}                                                         & \textit{-0.05}               & 0.87                                  \\ \hline
	\end{tabular}
	\caption{A quantitative comparison of uncertainty estimation methods on 1D datasets}
	\label{tab_quant_compare1D}
\end{table}
\begin{itemize}
	\item It can be inferred from the tabulation that Gaussian processes outperform the considered pair of uncertainty estimation methods in terms of every metric(values marked in bold). However, the pair GP models are included in this comparison only as a baseline.
	\item When it comes to comparing DER with MCDO(italicized values correspond to better performance), the former performs better in terms of NLL and EVA while the latter has relatively lower values of RMSE, the measure of predictive accuracy.
	\item Better performance of DER over MCDO\_ADF can be attributed to three important factors:
	\begin{itemize}
		\item Flexibility: The extent of confidence interval around a prediction estimated by a method.
		\item Quality of model fit to data
		\item Performance in Out-Of-Distribution regions(discussed more in the next section).
	\end{itemize}
	\item MCDO\_ADF suffers from high values of NLL (undesirable) in case of all the three functions due to lack of flexibility.
\end{itemize}
\subsubsection{Qualitative comparison}
\begin{itemize}
	\item As it can be witnessed from \ref{der_fn1},\ref{der_fn2} and \ref{der_fn3} DER provides a constant estimate for the value of aleatoric uncertainty similar to homoscedastic GP, which is desirable. 
	\item The MCDO\_ADF technique  performs relatively better than DER in estimating epistemic uncertainties for fn\_1(refer \ref{mcdo_fn1}), while its performance plummets in the case of fn\_3(refer \ref{mcdo_fn3}) where it fails to report high values of uncertainty for the OOD region which results in a very high value of NLL.
	\item The MCDO\_ADF variant of the neural network model performs better than its DER counterpart in modeling smoothness of functions. While in the case of GPs, the use of squared-exponential kernels as priors leads to better modeling of such functions.  
	\item The success of homoscedastic GPs in modeling  aleatoric uncertainties can be attributed to constancy in variance of Gaussian distribution used for adding noise to input data generated from a given function. 
	
	
\end{itemize}
\begin{figure}[H]
	\centering
	\begin{subfigure}[b]{0.4\textwidth}
		\centering
		\includegraphics[width=\textwidth]{toy_dataset/homo_fn1}
		\caption{Homoscedastic GP on fn\_1: $y=\frac{sin(3x)}{3x}$}
		\label{homo_fn1}
	\end{subfigure}
	\hfill
	\begin{subfigure}[b]{0.4\textwidth}
		\centering
		\includegraphics[width=\textwidth]{toy_dataset/hetero_fn1}
		\caption{Heteroscedastic GP on fn\_1: $y=\frac{sin(3x)}{3x}$}
		\label{hetero_fn1}
	\end{subfigure}
	\hfill
	\begin{subfigure}[b]{0.4\textwidth}
		\centering
		\includegraphics[width=\textwidth]{toy_dataset/mcdo_fn1}
		\caption{MCDO\_ADF on fn\_1: $y=\frac{sin(3x)}{3x}$}
		\label{mcdo_fn1}
	\end{subfigure}
	\hfill
	\begin{subfigure}[b]{0.4\textwidth}
		\centering
		\includegraphics[width=\textwidth]{toy_dataset/der_fn1}
		\caption{DER on fn\_1: $y=\frac{sin(3x)}{3x}$}
		\label{der_fn1}
	\end{subfigure}
	\hfill
	\begin{subfigure}[b]{0.4\textwidth}
		\centering
		\includegraphics[width=\textwidth]{toy_dataset/homo_fn2}
		\caption{Homoscedastic GP on fn\_2: $y=0.1x^3$}
		\label{homo_fn2}
	\end{subfigure}
	\hfill
	\begin{subfigure}[b]{0.4\textwidth}
		\centering
		\includegraphics[width=\textwidth]{toy_dataset/hetero_fn2}
		\caption{Heteroscedastic GP on fn\_2: $y=0.1x^3$}
		\label{hetero_fn2}
	\end{subfigure}
\end{figure}

\begin{figure}[H]\ContinuedFloat
	\centering
	\begin{subfigure}[b]{0.4\textwidth}
		\centering
		\includegraphics[width=\textwidth]{toy_dataset/mcdo_fn2}
		\caption{MCDO\_ADF on fn\_2: $y=0.1x^3$}
		\label{mcdo_fn2}
	\end{subfigure}
	\hfill
	\begin{subfigure}[b]{0.4\textwidth}
		\centering
		\includegraphics[width=\textwidth]{toy_dataset/der_fn2}
		\caption{DER on fn\_2: $y=0.1x^3$}
		\label{der_fn2}
	\end{subfigure}
	\hfill
	\begin{subfigure}[b]{0.4\textwidth}
		\centering
		\includegraphics[width=\textwidth]{toy_dataset/homo_fn3}
		\caption{Homoscedastic GP on fn\_3: $y=-(1+x)sin(1.2x)$}
		\label{homo_fn3}
	\end{subfigure}
	\hfill
	\begin{subfigure}[b]{0.4\textwidth}
		\centering
		\includegraphics[width=\textwidth]{toy_dataset/hetero_fn3}
		\caption{Heteroscedastic GP on fn\_3: $y=-(1+x)sin(1.2x)$}
		\label{hetero_fn3}
	\end{subfigure}
	\hfill
	\begin{subfigure}[b]{0.4\textwidth}
		\centering
		\includegraphics[width=\textwidth]{toy_dataset/mcdo_fn3}
		\caption{MCDO\_ADF on fn\_3: $y=-(1+x)sin(1.2x)$}
		\label{mcdo_fn3}
	\end{subfigure}
	\hfill
	\begin{subfigure}[b]{0.4\textwidth}
		\centering
		\includegraphics[width=\textwidth]{toy_dataset/der_fn3}
		\caption{DER on fn\_3: $y=-(1+x)sin(1.2x)$}
		\label{der_fn3}
	\end{subfigure}
	\hfill
	\caption{Plots depicting application of uncertainty estimation methods on 1D datasets}
	\label{fig_1d_functions}
\end{figure}
\end{document}

\section{Out-Of-Distribution(OOD) testing}
One of the important uses for having an estimate of uncertainty associated with a Neural Net model's output is \enquote{selective prediction}. This means that the confidence estimate can be used to determine the correctness of an output and in turn decide whether to consider it for further processing/decision-making or not. It is crucial for the uncertainty estimation method to produce a higher value of uncertainty when the model faces test samples from an unknown data distribution. This section evaluates the considered uncertainty estimation methods on their response to out-of-distribution inputs.
\subsection {Response to Out-Of-Distribution data}
\subsubsection{Steering angle dataset}

\subsubsection{1D datasets}
Train and test data ranges of the set of 1D functions are set in such a way that the Out-Of-Distribution (OOD) regions lie both in-between and on either sides of training data ranges. Owing to the smooth nature of target functions and use of squared exponential kernels GPs perform well both in terms of predictive accuracy and NLL(uncertainty estimation quality) in OOD areas. Both the considered uncertainty estimation methods do not perform well in predicting the mean and estimating uncertainties in OOD regions of fn\_3. Due to narrow confidence intervals proposed by MCDO\_ADF in this region, the method performs poorly in terms of NLL. In the case of fn\_2, predictions of both the methods remain close to the target. The MCDO\_ADF method outperforms DER by reporting a high value of epistemic uncertainty for OOD regions in fn\_1. 
\subsection{Response to adversarial attacks(for steering angle dataset only)}
Adversarial examples are malicious inputs designed to fool machine learning models[\cite{kurakin2016adversarial}]. Such data samples can be considered as an extreme case of Out-Of-Distribution as they are synthesized by perturbing inputs in an adversarial fashion to cause maximum error on the model prediction. An uncertainty estimation method needs to be capable of identifying and reporting an adversarially perturbed input data sample by producing a high value of uncertainty. 

In order to evaluate responses of the considered uncertainty estimation methods, samples from training data distribution are perturbed using the FGSM(Fast Gradient Sign Method)\cite{goodfellow2015explaining} and fed to models. A set of images produced from a given image subject to different levels of adversarial perturbations is shown in the figure\ref{fig_adv_example}. Adversarial images fed to a given Dronet model variant are synthesized using perturbations generated by its own.  

\begin{figure}[H]
	\centering
	\begin{subfigure}[b]{0.19\textwidth}
		\centering
		\includegraphics[width=\textwidth]{adv_0}
		\caption{$\epsilon=0$}
		\label{fig:y equals x}
	\end{subfigure}
	\hfill
	\begin{subfigure}[b]{0.19\textwidth}
		\centering
		\includegraphics[width=\textwidth]{adv_1}
		\caption{$\epsilon=0.02$}
		\label{fig:three sin x}
	\end{subfigure}
	\hfill
	\begin{subfigure}[b]{0.19\textwidth}
		\centering
		\includegraphics[width=\textwidth]{adv_2}
		\caption{$\epsilon=0.06$}
		\label{fig:five over x}
	\end{subfigure}
	\hfill
	\begin{subfigure}[b]{0.19\textwidth}
		\centering
		\includegraphics[width=\textwidth]{adv_3}
		\caption{$\epsilon=0.08$}
		\label{fig:five over x}
	\end{subfigure}
	\hfill
	\begin{subfigure}[b]{0.19\textwidth}
		\centering
		\includegraphics[width=\textwidth]{adv_4}
		\caption{$\epsilon=0.10$}
		\label{fig:five over x}
	\end{subfigure}
	\caption{Increasing levels of adversarial noise}
	\label{fig_adv_example}
\end{figure}

\begin{figure}[H]
	\centering
	\includegraphics[scale=0.40]{adversarial/adv_plots}
	\caption{Plots of $\epsilon$(adversarial perturbations) against uncertainties and RMSE}
	\label{fig_adv_analysis}
\end{figure}



Plots in the figure \ref{fig_adv_analysis} depict the impact of increasing perturbation levels(denoted by $\epsilon$) on average values of uncertainty estimates and RMSE values respectively, for both model variants. The following can be inferred from the set of plots:
\begin{itemize}
	\item The existence of linear relationships between $\epsilon$ , RMSE and total variance respectively is desirable and indicates that the MCDO\_ADF model variant is better calibrated(alignment between error and uncertainty) than its DER counterpart.
	\item  Though aleatoric uncertainties estimated by DER are more sensitive to increasing levels of $\epsilon$ than MCDO\_ADF, the method's response is non-linear in nature.
	\item  Epistemic uncertainty forms the major component of total uncertainty estimated by MCDO\_ADF while the aleatoric component dominates in the case of DER.
\end{itemize}
\subsubsection{Spearman's correlation analysis}

Spearman's rank correlation (often denoted by $\rho$) is a measure of statistical dependence between a pair of variables. Intuitively $\rho$ indicates the strength and direction of association between two ranked variables. $\rho$ varies between -1 and 1. The direction of association is indicated by sign of $\rho$ while its magnitude indicates the strength of correlation between the variable pair.

In the context of analyzing response of uncertainty estimation methods to adversarial attacks, Spearman's rank correlation coefficient is computed for pairs of following variables : $\epsilon$, absolute deviation of a prediction from its corresponding ground truth value, aleatoric, epistemic and total uncertainties. Though Pearson's correlation coefficient is a common choice for correlation analysis, there are two reasons to prefer Spearman's correlation for this analysis:
\begin{itemize}
	\item Pearson's correlation measure applies to normally distributed variables. As no assumptions are made on underlying distributions of the considered set of variables Spearman's correlation coefficient is preferred.
	\item Spearman's coefficient measures correlation between a pair of ranked variables. As this analysis focuses on effects of increasing levels of $\epsilon$ (a ranked variable) on other variables, the measure becomes an apt choice.
\end{itemize} 
Spearman's correlation coefficients are arranged in form of a symmetric matrix (indices denoting variables) separately for both MCDO\_ADF and DER model variants as shown in the figure below.
\begin{figure}[H]
	\begin{subfigure}[b]{0.55\textwidth}
		\includegraphics[width=\textwidth]{adversarial/evi_correlation_matrix}
		\caption{Spearman's correlation matrix for DER}
		\label{fig:five over x}
	\end{subfigure}
	\hfill
	\begin{subfigure}[b]{0.55\textwidth}
		\includegraphics[width=\textwidth]{adversarial/mcdo_adf_correlation_matrix}
		\caption{Spearman's correlation matrix for MCDO\_ADF}
		\label{fig:five over x}
	\end{subfigure}
	\caption{Spearman's correlation heatmaps}
	\label{fig_correlation_analysis}
\end{figure}
The following can be inferred from the computed correlation coefficient matrices:
\begin{itemize}
	\item MCDO\_ADF shows relatively a stronger correlation between $\epsilon$ and other variables of interest than DER. This indicates a higher-level of alignment between $\epsilon$ and uncertainties estimated by MCDO\_ADF, which is desirable.
	\item A stronger association exists between epistemic and aleatoric components of uncertainty estimated by DER ($\rho = 0.97$) than MCDO\_ADF($\rho = 0.89$). This can be attributed to DER's ability to relate the pair of uncertainty components.
	\item The dominance of aleatoric and epistemic uncertainty components in total uncertainties estimated by DER and MCDO\_ADF respectively can observed in their corresponding correlation coefficients. This aligns with inferences obtained from \ref{fig_adv_analysis}. 
\end{itemize}

