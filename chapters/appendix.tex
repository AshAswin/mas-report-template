%!TEX root = ../report.tex

\begin{document}
    \chapter{Metrics}
    An extensive review was conducted to find a suitable metric that overcomes limitations of Negative-Log-Likelihood (NLL) in capturing the quality of approximated posterior distribution. The following metrics were considered for analysis:
    \begin{itemize}
	\item Brier score
	\item Expected Calibration Error
	\item Maximum Calibration Error
	\item Bayesian Information Criterion
	\item Mutual Information
	\item Entropy
	\item Cross Entropy
	\item Kullback-Leibler Divergence
	\item Signal to Noise Ratio
    \end{itemize}
	Though some of them suit for measuring calibration, sharpness and uncertainty estimation quality in a classification setup, none of them apply to regression due to the continuous and unbounded nature of its output space. Understanding and approaching Deep Neural Networks from the perspective of Information theory \cite{saxe2019on} may prove helpful in devising a suitable metric to measure the quality of uncertainty estimates in the regression setup.
	\chapter{Correlation matrices for OOD Analysis}
	The following are Spearman's correlation coefficient matrices for MCDO\_ADF and DER methods, obtained by relating variables involved in the extended OOD response analysis described in \ref{ood_extended}. Apart from the strong correlation value between darkness levels (darkness\_coefficient) and prediction error (abs\_deviation), values of Spearman's coefficient for other pairs of variables do not mean much as they do not have a monotonic relationship (refer Figure \ref{fig_ood_extended}), which is an important criterion for analysing a problem using Spearman's correlation analysis.
	\begin{figure}[H]
	\begin{subfigure}[b]{0.5\textwidth}
	\includegraphics[width=1\textwidth,right]{ood/mcdo_adf_correlation_matrix}
	\caption{Spearman's correlation matrix-MCDO\_ADF}
	\label{homo_fn1}
	\end{subfigure}
	\hfill
	\begin{subfigure}[b]{0.5\textwidth}
	\includegraphics[width=1\textwidth,right]{ood/evi_correlation_matrix}
	\caption{Spearman's correlation matrix-DER}
	\label{hetero_fn1}
	\end{subfigure}
	\hfill
	\end{figure}
\end{document}
